\documentclass[a4paper]{article} %四号=14pt,小四=12pt,五号=10.5pt
\usepackage[heading=true]{ctex} %设置中文章节标题
\usepackage{graphicx} %插图
\usepackage{float} % 使用H控制浮动
\usepackage[section]{placeins} %限制浮动范围
\usepackage{booktabs} %使用三线表
\usepackage{listings} %使用环境lstlisting
\usepackage{minted} %使用环境minted排程序更美观,需要python包Pygments
\usepackage{xcolor} %使用color

\usepackage{amsmath}
\usepackage{amssymb} %mathbb
\usepackage{amsthm} %proof
\usepackage[margin=1in]{geometry} %调整页边距与word相同
\usepackage{algorithm}
% \usepackage{algorithmicx}
\usepackage{algpseudocode}
\usepackage[colorlinks=true]{hyperref} %超链接,为减少冲突,需要将其放在其他宏包之后。
\numberwithin{equation}{section} %按节编号
% \setcounter{tocdepth}{3}

\newcommand{\degree}{^\circ}
\newcommand{\ud}{\,\mathrm{d}}

\newtheorem{Theorem}{定理}

\title{\heiti \LaTeX\; Cookbook}
\author{\kaishu 余周炜\thanks{Email: \href{mailto:yuzhouwei6326@outlook.com}{yuzhouwei6326@outlook.com}}}
\date{\today}
\begin{document}
% \frontmatter
\maketitle

\begin{abstract}
本文用简短的语言让你快速上手LaTeX。
\end{abstract}

\clearpage
\tableofcontents

% \mainmatter
\newpage
\section{\TeX 须知}\label{sec:tex}

\subsection{概述}
\TeX\ 是高德纳(Donald E.Knuth)开发的、以排版文字和数学公式为目的的一个计算机软件。

\subsection{查看宏包}

在Terminal下输入下列命令:
\begin{lstlisting}
    texdoc <pkg-name>
\end{lstlisting}

可以参看包的说明。第\ref{sec:tex}节的页码是\pageref{sec:tex}。

查看常用的数学符号和公式参见\href{https://katex.org/docs/supported.html}{KaTeX Supported Functions}。

查看常用的宏包和用法参见\href{https://en.wikibooks.org/wiki/LaTeX}{WIKIBOOKS \LaTeX}有关内容。

\section{符号和字体}
This is my \emph{first} document prepared in \LaTeX. I typed it on \today.
The complete \TeX\ reserved charcaters are \textasciitilde, \#, \$, \%, 
\textasciicircum, \&, \_, \textbackslash,\{,\}. \\[10pt]
\noindent This is the second line.Here are some font families:
\textrm{roman}, \textsf{sans serif}, \texttt{typewriter}, \textmd{medium}, 
\textbf{boldface}, \textup{upright}, \textit{italic}, \textsl{slanted}, 
\textsc{small cap}.
\subsection{Size}\label{sub:size}
\subsubsection{size} \label{ssb:size}
如\ref{sub:size}小节\ref{ssb:size}子节所示:{\tiny size} {\small size} 
{\large size} {\Large size} {\LARGE size} {\huge size} {\Huge size}

\begin{minted}{tex}
This is my \emph{first} document prepared in \LaTeX. I typed it on \today.
The complete \TeX\ reserved charcaters are \textasciitilde, \#, \$, \%, 
\textasciicircum, \&, \_, \textbackslash,\{,\}. \\[10pt]
\noindent This is the second line.
\textrm{roman}, \textsf{sans serif}, \texttt{typewriter}, \textmd{medium}, 
\textbf{boldface}, \textup{upright}, \textit{italic}, \textsl{slanted}, 
\textsc{small cap}.
\subsection{Size}\label{sub:size}
\subsubsection{size} \label{ssb:size}
如\ref{sub:size}小节\ref{ssb:size}子节所示:{\tiny size} {\small size} 
{\large size} {\Large size} {\LARGE size} {\huge size} {\Huge size}
\end{minted}

\section{文档}

\begin{quote}
“天地玄黄,宇宙洪荒。日月盈昃,辰宿列张。”\footnote{出自《千字文》}。
\end{quote}

\subsection{有序列表}
\begin{minipage}{0.5\textwidth}
\begin{enumerate}
\item An item.
\begin{enumerate}
\item A nested item.
\item[*] A starred item.
\item Another item. \label{itm:itref}
\end{enumerate}
\item Go back to upper level.
\item Reference(\ref{itm:itref}).
\end{enumerate}
\end{minipage}
\begin{minipage}{0.5\textwidth}
\begin{minted}{tex}
\begin{enumerate}
\item An item.
\begin{enumerate}
\item A nested item.
\item[*] A starred item.
\item Another item. \label{itm:itref}
\end{enumerate}
\item Go back to upper level.
\item Reference(\ref{itm:itref}).
\end{enumerate}
\end{minted}
\end{minipage}

\subsection{无序列表}

\begin{minipage}{0.5\textwidth}
\begin{itemize}
\item An item.
\begin{itemize}
\item A nested item.
\item[+] A ‘plus’ item.
\item Another item.
\end{itemize}
\item Go back to upper .
\end{itemize}
\end{minipage}
\begin{minipage}{0.5\textwidth}
\begin{minted}{tex}
\begin{itemize}
\item An item.
\begin{itemize}
\item A nested item.
\item[+] A ‘plus’ item.
\item Another item.
\end{itemize}
\item Go back to upper.
\end{itemize}
\end{minted}
\end{minipage}

\subsection{对齐}
\begin{center}
Centered text using a
\verb|center| environment.等价于\verb|\centering|
\end{center}
\begin{flushleft}
Left-aligned text using a
\verb|flushleft| environment.%verb原文照排
\end{flushleft}
\begin{flushright}
Right-aligned text using a
\verb|flushright| environment.
\end{flushright}

\subsection{表格}
\begin{table}[htbp]
\centering
\begin{tabular}{p{5cm}ccc}
\toprule 
& \multicolumn{3}{c}{Numbers} \\
\cmidrule{2-4} 
& 1 & 2 & 3 \\
\midrule
Alphabet & A & B & C \\
Roman & I & II& III \\
\bottomrule 
\end{tabular}
\caption{三线表} 
\end{table}

\begin{minted}{tex}
    \begin{table}[htbp]
    \centering
    \begin{tabular}{p{5cm}ccc}
    \toprule 
    & \multicolumn{3}{c}{Numbers} \\
    \cmidrule{2-4} 
    & 1 & 2 & 3 \\
    \midrule
    Alphabet & A & B & C \\
    Roman & I & II& III \\
    \bottomrule 
    \end{tabular}
    \caption{三线表} 
    \end{table}
\end{minted}

\subsection{图片}
\begin{figure}[H]
\centering
\includegraphics[width=0.2\textwidth]{tongji.jpg}
\caption{同济校徽}
\end{figure}

\begin{minted}{tex}
    \begin{figure}[H]
    \centering
    \includegraphics[width=0.2\textwidth]{tongji.jpg}
    \caption{同济校徽}
    \end{figure}
\end{minted}

\subsection{盒子}
\begin{minipage}[t]{0.5\textwidth}
千字文:\\
天地玄黄 宇宙洪荒。\footnote{脚注来自minipage}
\end{minipage}
\begin{minipage}{0.5\textwidth}
\begin{center}
\bfseries 西江月$\cdot$ 证明
\end{center}
\hspace{2em}即得易见平凡,仿照上例显然。留作习题答案略,读者自证不难。

\hspace{2em}反之亦然同理,推论自然成立。略去过程QED,由上可知证毕。
\end{minipage}
\vspace{2cm}

\textsf{文字用\textcolor{red}{红色}强调\\
\colorbox{gray}{深灰色背景} 
\colorbox[gray]{0.95}{浅灰色背景} \\
\fcolorbox{blue}{yellow}{%
\textcolor{blue}{蓝色边框+文字,黄色背景}}
}


\section{数学公式}

Add $a$ squared and $b$ squared to get $c$ squared
\begin{equation}
a^2+b^2=c^2 \label{eq:gougu}
\end{equation}

称公式\eqref{eq:gougu}为勾股定理。

行内公式:
$\lim_{n \to \infty}
\sum_{k=1}^n \frac{1}{k^2}
= \frac{\pi^2}{6}$
的显示较行间公式局促:
\begin{equation*}
\lim_{n \to \infty}
\sum_{k=1}^n \frac{1}{k^2}
= \frac{\pi^2}{6}
\end{equation*}

加\verb|\displaystyle|可以改变这一情况:$\displaystyle \lim_{n \to \infty}
\sum_{k=1}^n \frac{1}{k^2}
= \frac{\pi^2}{6}$

公式和字符混排:
\begin{equation}x^{2} \ge 0 \qquad \forall
\text{(for \textbf{all})}
x\in\mathbb{R}
\end{equation}

Pascal's rule is
\begin{equation}
\binom{n}{k} =\binom{n-1}{k}
+ \binom{n-1}{k-1}
\end{equation}

叠加:
\[
f_n(x) \stackrel{*}{\approx} 1
\]

使用\verb|\left| 和\verb|\right| 命令可令括号(定界符)的大小可变,在行间公式中常用。\LaTeX 会自动根据括号内的公式大小决定定界符大小。\verb|\left| 和\verb|\right| 必须成对使用。需要使用单个定界符时,另一个定界符写成 \verb|\left.| 或 \verb|\right.|

对齐:
\begin{align}
a =& b + c \notag \\
=& d + e
\end{align}

不对齐:
\begin{gather}
a = b + c \\
d = e + f + g \\
h + i = j + k \notag \\
l + m = n
\end{gather}

矩阵:
\begin{equation}
\mathbf X = 
\begin{pmatrix}
x_{11} & x_{12} & \ldots & x_{1n}\\
x_{21} & x_{22} & \ldots & x_{2n}\\
\vdots & \vdots & \ddots & \vdots\\
x_{n1} & x_{n2} & \ldots & x_{nn}\\
\end{pmatrix}
\end{equation}

选择:
\begin{equation}
|x| =
\begin{cases}
-x & \text{if } x < 0,\\
0 & \text{if } x = 0,\\
x & \text{if } x > 0.
\end{cases}
\end{equation}

积分的写法:
\begin{equation}
\int_a^b f(x)\ud x
\end{equation}

定理:
\begin{Theorem}[Energy]
The relationship of energy,
momentum and mass is
\begin{equation}
E^2 = m_0^2 c^4 + p^2 c^2
\end{equation}
\end{Theorem}

证明\cite[page 48]{lshort}:
\begin{proof}
For simplicity, we use
\[
E=mc^2 \qedhere
\]
\end{proof}

\section{代码和算法}
\subsection{代码}
\begin{minted}[linenos]{c++}
#include <iostream>
using namespace std;
int main() 
{
    cout << "Hello, World!";
    return 0;
}
\end{minted}

\subsection{算法}
\begin{algorithm}[H]
\caption{Euclid’s algorithm}\label{alg:euclid}
\begin{algorithmic}[1] %number every line, 0:no line numbering
\Procedure{Euclid}{$a,b$}\Comment{The g.c.d. of a and b}
\State $r\gets a\bmod b$
\While{$r\not=0$}\Comment{We have the answer if r is 0}
\State $a\gets b$
\State $b\gets r$
\State $r\gets a\bmod b$
\EndWhile\label{alg:euclidendwhile}
\State \textbf{return} $b$\Comment{The gcd is b}
\EndProcedure
\end{algorithmic}
\end{algorithm}

The \textbf{while} in algorithm
\ref{alg:euclid} ends in line
\ref{alg:euclidendwhile}, so
\algref{alg:euclid}{alg:euclidendwhile}
is the line we seek.


\begin{algorithm}[H]
\caption{Part 1}
\begin{algorithmic}[1]
\Procedure {BellmanKalaba}{$G$, $u$, $l$, $p$}
\ForAll {$v \in V(G)$}
\State $l(v) \leftarrow \infty$
\EndFor
\State $l(u) \leftarrow 0$
\Repeat
\For {$i \leftarrow 1, n$}
\State $min \leftarrow l(v_i)$
\For {$j \leftarrow 1, n$}
\If {$min > e(v_i, v_j) + l(v_j)$}
\State $min \leftarrow e(v_i, v_j) + l(v_j)$
\State \Comment For some reason we need to break here!
\algstore{bkbreak}
\end{algorithmic}
\end{algorithm}

    And we need to put some additional text between\dots

\begin{algorithm}[H]
\caption{Part 2}
\begin{algorithmic}[1]
\algrestore{bkbreak}
\State $p(i) \leftarrow v_j$
\EndIf
\EndFor
\State $l’(i) \leftarrow min$
\EndFor
\State $changed \leftarrow l \not= l’$
\State $l \leftarrow l’$
\Until{$\neg changed$}
\EndProcedure
\end{algorithmic}
\end{algorithm}

\appendix
\section{版权}

未经作者允许不得随意转载,需要转载请联系作者,邮箱:\\\href{mailto:yuzhouwei6326@outlook.com}{yuzhouwei6326@outlook.com}。

\section{部分源代码}

\begin{minted}[linenos]{tex}
    \documentclass[a4paper]{article} %四号=14pt,小四=12pt,五号=10.5pt
    \usepackage[heading=true]{ctex} %设置中文章节标题
    \usepackage{graphicx} %插图
    \usepackage{float} % 使用H控制浮动
    \usepackage[section]{placeins} %限制浮动范围
    \usepackage{booktabs} %使用三线表
    \usepackage{listings} %使用环境lstlisting
    \usepackage{minted} %使用环境minted排程序更美观,需要python包Pygments
    \usepackage{xcolor} %使用color
    
    \usepackage{amsmath}
    \usepackage{amssymb} %mathbb
    \usepackage{amsthm} %proof
    \usepackage[margin=1in]{geometry} %调整页边距与word相同
    \usepackage{algorithm}
    % \usepackage{algorithmicx}
    \usepackage{algpseudocode}
    \usepackage[colorlinks=true]{hyperref} %超链接,为减少冲突,需要将其放在其他宏包之后。
    \numberwithin{equation}{section} %按节编号
    % \setcounter{tocdepth}{3}
    
    \newcommand{\degree}{^\circ}
    \newcommand{\ud}{\,\mathrm{d}}
    
    \newtheorem{Theorem}{定理}
    
    \title{\heiti \LaTeX\; Cookbook}
    \author{\kaishu 余周炜\thanks{Email: 
    \href{mailto:yuzhouwei6326@outlook.com}{yuzhouwei6326@outlook.com}}}
    \date{\today}
    \begin{document}
    % \frontmatter
    \maketitle
    
    \begin{abstract}
    本文用简短的语言让你快速上手LaTeX。
    \end{abstract}
    
    \clearpage
    \tableofcontents
    
    % \mainmatter
    \newpage
    \section{\TeX 须知}\label{sec:tex}
    
    \subsection{概述}
    \TeX\ 是高德纳(Donald E.Knuth)开发的、以排版文字和数学公式为目的的一个计算机软件。
    \subsection{查看宏包}
    在Terminal下输入下列命令:
    \begin{lstlisting}
        texdoc <pkg-name>
    \end{lstlisting} 
    可以参看包的说明。第\ref{sec:tex}节的页码是\pageref{sec:tex}。
    
    查看常用的数学符号和公式参见
    \href{https://katex.org/docs/supported.html}{KaTeX Supported Functions}。
    
    查看常用的宏包和用法参见
    \href{https://en.wikibooks.org/wiki/LaTeX}{WIKIBOOKS \LaTeX}有关内容。
    
    \section{符号和字体}    
    \section{文档}
    \begin{quote}
    “天地玄黄,宇宙洪荒。日月盈昃,辰宿列张。”\footnote{出自《千字文》}。
    \end{quote}

    \subsection{对齐}
    \begin{center}
    Centered text using a
    \verb|center| environment.等价于\verb|\centering|
    \end{center}
    \subsection{盒子}
    \begin{minipage}[t]{0.5\textwidth}
    千字文:\\
    天地玄黄 宇宙洪荒。\footnote{脚注来自minipage}
    \end{minipage}
    三字经:\parbox[t]{3em}%
    {人之初 性本善 性相近 习相远}
    \quad
    \vspace{2cm}
    \textsf{文字用\textcolor{red}{红色}强调\\
    \colorbox{gray}{深灰色背景} 
    \colorbox[gray]{0.95}{浅灰色背景} \\
    \fcolorbox{blue}{yellow}{%
    \textcolor{blue}{蓝色边框+文字,黄色背景}}
    }
    
    \section{数学公式}
    Add $a$ squared and $b$ squared to get $c$ squared
    \begin{equation}
    a^2+b^2=c^2 \label{eq:gougu}
    \end{equation}
    称公式\eqref{eq:gougu}为勾股定理。
    
    行内公式:
    $\lim_{n \to \infty}
    \sum_{k=1}^n \frac{1}{k^2}
    = \frac{\pi^2}{6}$
    的显示较行间公式局促:
    \begin{equation*}
    \lim_{n \to \infty}
    \sum_{k=1}^n \frac{1}{k^2}
    = \frac{\pi^2}{6}
    \end{equation*}
    
    加\verb|\displaystyle|可以改变这一情况:$\displaystyle \lim_{n \to \infty}
    \sum_{k=1}^n \frac{1}{k^2}
    = \frac{\pi^2}{6}$
    
    公式和字符混排:
    \begin{equation}x^{2} \ge 0 \qquad \forall
    \text{(for \textbf{all})}
    x\in\mathbb{R}
    \end{equation}
    
    Pascal's rule is
    \begin{equation}
    \binom{n}{k} =\binom{n-1}{k}
    + \binom{n-1}{k-1}
    \end{equation}
    
    叠加:
    \[
    f_n(x) \stackrel{*}{\approx} 1
    \]
    
    使用\verb|\left| 和\verb|\right| 命令可令括号(定界符)的大小可变,
    在行间公式中常用。\LaTeX 会自动根据括号内的公式大小决定定界符大小。
    \verb|\left| 和\verb|\right| 必须成对使用。需要使用单个定界符时,
    另一个定界符写成 \verb|\left.| 或 \verb|\right.|
    
    对齐:
    \begin{align}
    a =& b + c \notag \\
    =& d + e
    \end{align}
    
    不对齐:
    \begin{gather}
    a = b + c \\
    d = e + f + g \\
    h + i = j + k \notag \\
    l + m = n
    \end{gather}
    
    矩阵:
    \begin{equation}
    \mathbf X = 
    \begin{pmatrix}
    x_{11} & x_{12} & \ldots & x_{1n}\\
    x_{21} & x_{22} & \ldots & x_{2n}\\
    \vdots & \vdots & \ddots & \vdots\\
    x_{n1} & x_{n2} & \ldots & x_{nn}\\
    \end{pmatrix}
    \end{equation}
    
    选择:
    \begin{equation}
    |x| =
    \begin{cases}
    -x & \text{if } x < 0,\\
    0 & \text{if } x = 0,\\
    x & \text{if } x > 0.
    \end{cases}
    \end{equation}
    
    积分的写法:
    \begin{equation}
    \int_a^b f(x)\ud x
    \end{equation}
    
    定理:
    \begin{Theorem}[Energy]
    The relationship of energy,
    momentum and mass is
    \begin{equation}
    E^2 = m_0^2 c^4 + p^2 c^2
    \end{equation}
    \end{Theorem}
    
    证明\cite[page 48]{lshort}:
    \begin{proof}
    For simplicity, we use
    \[
    E=mc^2 \qedhere
    \]
    \end{proof}
    
    \section{代码和算法}
    \subsection{代码}
    \begin{minted}[linenos]{c++}
    #include <iostream>
    using namespace std;
    int main() 
    {
        cout << "Hello, World!";
        return 0;
    }
    \end{minte}d %为了不提早结束
    
    \subsection{算法}
    \begin{algorithm}[H]
    \caption{Euclid’s algorithm}\label{alg:euclid}
    \begin{algorithmic}[1] %number every line, 0:no line numbering
    \Procedure{Euclid}{$a,b$}\Comment{The g.c.d. of a and b}
    \State $r\gets a\bmod b$
    \While{$r\not=0$}\Comment{We have the answer if r is 0}
    \State $a\gets b$
    \State $b\gets r$
    \State $r\gets a\bmod b$
    \EndWhile\label{alg:euclidendwhile}
    \State \textbf{return} $b$\Comment{The gcd is b}
    \EndProcedure
    \end{algorithmic}
    \end{algorithm}
    
    The \textbf{while} in algorithm
    \ref{alg:euclid} ends in line
    \ref{alg:euclidendwhile}, so
    \algref{alg:euclid}{alg:euclidendwhile}
    is the line we seek.
    
    \begin{algorithm}[H]
    \caption{Part 1}
    \begin{algorithmic}[1]
    \Procedure {BellmanKalaba}{$G$, $u$, $l$, $p$}
    \ForAll {$v \in V(G)$}
    \State $l(v) \leftarrow \infty$
    \EndFor
    \State $l(u) \leftarrow 0$
    \Repeat
    \For {$i \leftarrow 1, n$}
    \State $min \leftarrow l(v_i)$
    \For {$j \leftarrow 1, n$}
    \If {$min > e(v_i, v_j) + l(v_j)$}
    \State $min \leftarrow e(v_i, v_j) + l(v_j)$
    \State \Comment For some reason we need to break here!
    \algstore{bkbreak}
    \end{algorithmic}
    \end{algorithm}
    
        And we need to put some additional text between\dots
    
    \begin{algorithm}[H]
    \caption{Part 2}
    \begin{algorithmic}[1]
    \algrestore{bkbreak}
    \State $p(i) \leftarrow v_j$
    \EndIf
    \EndFor
    \State $l’(i) \leftarrow min$
    \EndFor
    \State $changed \leftarrow l \not= l’$
    \State $l \leftarrow l’$
    \Until{$\neg changed$}
    \EndProcedure
    \end{algorithmic}
    \end{algorithm}
    
    \appendix
    \section{源代码}
    % \backmatter
    \begin{thebibliography}{99}
    \bibitem{lshort}Tobias Oetiker:一份不太简短的\LaTeXe 介绍。
    \bibitem{ltxprimer}Trivandrum: \LaTeX\ Tutorials: A Primer, India。
    \end{thebibliography}
    \end{document}
\end{minted}

% \backmatter
\begin{thebibliography}{99}
\bibitem{lshort}Tobias Oetiker:一份不太简短的\LaTeXe 介绍。
\bibitem{ltxprimer}Trivandrum: \LaTeX\ Tutorials: A Primer, India。
\end{thebibliography}


\end{document}